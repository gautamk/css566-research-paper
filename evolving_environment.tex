\section*{Evolving Environment}

The outcome of a well managed software organisation are products and knowledge which provide value to the customer and satisfy their need. In today's fast changing technology landscape the needs and expectations of the customers is always evolving. Such an evolution is affected by various factors, some of which are discussed in the following sections.

\subsection*{Changing customer expectations}

The first few generations of software customers were primarily educational institutions and enterprises where software was merely a means to an end. Customers today, especially in the consumer software market expect an experience rather than just a tool that solves a problem. This explains the reason why organisations today have roles for professionals who work towards streamlining the way people use software \cite{kolko_design_2015} \cite{dixon_stop_2010}.

Customers have also tuned to the idea that organisations which build software systems  are expected to support their products for a long period of time when compared to hardware manufacturers who can get away with launching a new version of their product every year and deprecating old products every 2 years. This could potentially explain the meteoric rise in value of customer support services the market for which is projected to be valued at \$81 billion \cite{wuyts_outsourcing_2015}.

These changes in the customer expectation landscape requires new strategies to be incorporated and verified within the software management process.

\subsection*{New technologies}

Virtual reality, Internet of Things and Machine learning are the hottest consumer technologies evolving as of this writing and all these new technologies promise a new dimension of interactivity and user experience \cite{bachvarov_design-by--customer_2009}. The promise of a new user experience and interactivity comes with added complexity and risk, for example Virtual reality requires new hardware and software capabilities which increases the potential chances of failure and development teams need to evolve new techniques of managing this risk.

\subsection*{Product life cycles and Business models}

Web 2.0 has made a significant impact on the way users and companies buy and sell software. The traditional model of expensive one-time licensing fee has evolved into a small monthly subscription fee and companies are expected to deliver constant updates to their software product. 

A move to the subscription model from the licensing model might require implementing a continuous delivery strategy which might involve simplifying and streamlining the CI system to perform a large number smaller builds rather than single large builds \cite{schief_transforming_2012} \cite{di_valentin_measuring_2012}.


\subsection*{Open source}

Open source software has had a significant impact on the way companies develop software. Open source software helps developers build software faster and enables tapping into the collective wisdom of a larger community of developers. This translates into large savings for companies which is why many organisations are embracing the trend of using open source components within their technology stack \cite{baldwin_4_2014} \cite{lemmens_open_2008}.

Introducing and integrating open source software isn't free, though there is no upfront licensing cost for using open source software. Using open source software requires dedicated effort to adapt to the structure of the software library in question. Security is another concern when using open source software so keeping the system up to date with patches becomes a mandate to preserve the privacy of users and also to fulfil certain legal requirements.

\subsection*{New generation of developers}

A whole new generation of developers have grown up embracing collaborative development techniques such as Q\&A sites, open source software and Hackathons and access or lack thereof to such a community within an organisation may have a significant impact on the productivity of developers \cite{vasilescu_continuous_2014}. 


\subsection*{Effect on Businesses and Software management}

The aforementioned factors can have a significant impact on the projects, people and incidentally the core business of a company. So to achieve business agility and quickly adapt to an ever changing landscape, companies which rely on software technologies need to be able to quickly adapt their software management process \cite{mathiassen_business_2006}.
