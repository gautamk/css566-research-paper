\documentclass[10pt,conference]{IEEEtran}
%\documentclass[letterpaper,twocolumn,10pt]{article}

\usepackage{graphicx}
\usepackage{url}
\usepackage[usenames]{color}
\usepackage{listings}
\usepackage[caption=false]{subfig}


%------------------------------------------------------------------------- 
% take the % away on next line to produce the final camera-ready version 
\pagestyle{empty}

%------------------------------------------------------------------------- 
\begin{document}

\title{Effective Continuous Integration in content: \newline Why writers need 
a CI}


%for single author (just remove % characters)
\author{
{\rm Gautam Kumar, Prof. David Socha}\\
Computing and Software Systems\\
University of Washington Bothell\\
gautamk@uw.edu, socha@gautamk.com
} % end author

\maketitle
\thispagestyle{empty}


\section*{Abstract}
Continuous Integration is considered an integral part of Agile development. 
CI is often thought to be a safety net that prevents developers from deploying 
broken code to production. 

Writing code is a creative exercise where developers organize instructions on 
a logical order for computers to execute. This process is not unlike writing 
content where authors organize information in a logical order for humans to 
understand. Developers work collaboratively and so do authors. Computer 
programs are expected to conform to a syntax and structure, while article are 
have rules on spelling, grammar and structure. Based on these similarities a 
CI system designed for writers and authors could potentially save hundreds of 
hours of work by an editor scrutinizing mundane details of an article.

This paper aims to analyze the potential benefits of authors adopting a CI 
from the perspective of a team collaboratively working on a single content 
writing project.

\section*{CI system description}
\section*{Methods of testing}
\subsection*{Testing Spelling}
\subsection*{Testing Grammar}
\subsection*{Testing Sentence structure}
\section*{Benefits and value}
\section*{Limitations and Practicality}
\section*{Conclusion}

\bibliographystyle{plain}
\bibliography{css566_software_management}

\end{document}

