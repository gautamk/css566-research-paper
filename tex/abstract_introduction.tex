\section*{Abstract}
Continuous Integration, A practice where developers integrate 
frequently \cite{stahl_modeling_2014} is considered an integral part of Agile 
development. CI is often thought to be a safety net that prevents developers from deploying broken code to production. 

A similar process and ideology of continual improvement and integration can be applied to Software Management, The art and science of planning and leading software projects \cite{stellman_applied_2005}.

This paper attempts to explore the process of creating such an integration practice within an Organisation.

\subsubsection*{Keywords}

Software Management, Agile development, Continuous Integration

\section*{Introduction}

Martin Fowler in his seminal article \cite{fowler_continuous_2006} on Continuous Integration describes it as a practice where members of a team integrate their work frequently and each integration is verified by an automated build to detect problems quickly.

In the context described above integration is the process of combining the work of all the developers of a project into a cohesive software. On the other hand if considered generically, Integration can be thought of as a practice of combining the work of multiple people and verifying its correctness. Using a similar collective, Continuous integration could be the process of merging the work of multiple people after specific events or set periods of time.