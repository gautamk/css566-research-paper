\section*{Evolving Environment}

In today's fast changing technology landscape, traditional definitions of continuous integration and improvement are being challenged to adapt quickly. In the following sections we discuss some potential factors which affect Continuous Integration deployments in an organisation.

\subsection*{Evolving Customer}
The previous generation of customers were primarily using a desktop computer with a reliable internet connection and a large display. This customer base is quickly moving towards mobile devices as their primary computing device where displays are much smaller, internet connection is spotty and there are restrictions on power consumption. 

These changes require the developers to adapt to the environment which inherently needs an overhaul of the software management process. \citationneeded


\subsection*{New technologies}

Virtual reality, Internet of Things and Machine learning are the hottest consumer technologies evolving as of this writing and all these new technologies promise a new dimension of interactivity and user experience. The promise of a new user experience and interactivity comes with added complexity and risk, for example Virtual reality requires new hardware and software capabilities which increases the potential chances of failure and development teams need to evolve new techniques of managing this risk. Traditional continuous integration many not work in this scenario. \citationneeded

\subsection*{Product life cycles and Business models}

Web 2.0 has made a significant impact on the way users and companies buy and sell software. The traditional model of expensive one-time licensing fee has evolved into a small monthly subscription fee and companies are expected to deliver constant updates to their software product. 


A move to the subscription model from the licensing model might require implementing a continuous delivery strategy which might involve simplifying and streamlining the CI system to perform a large number smaller builds rather than single large builds.

\citationneeded

\subsection*{Open source}

Open source software has had a significant impact on the way companies develop software. Open source software helps developers build software faster and enables tapping into the collective wisdom of a larger community of developers. This translates into large savings for companies which is why many organisations are embracing the trend of using open source components within their technology stack \cite{baldwin_4_2014} \cite{lemmens_open_2008}.

Introducing and integrating open source software isn't free, though there is no upfront licensing cost for using open source software. Using open source software requires dedicated effort to adapt to the structure of the software library in question. Security is another concern when using open source software so keeping the system up to date with patches becomes a mandate to preserve the privacy of users and also to fulfil certain legal requirements.

\subsection*{New generation of developers}

A whole new generation of developers have grown up embracing collaborative development techniques such as Q\&A sites, open source software and Hackathons and access or lack thereof to such a community within an organisation may have a significant impact on the productivity of developers \cite{vasilescu_continuous_2014}. 


\subsection*{Effect on Businesses and Software management}

The aforementioned factors can have a significant impact on the projects, people and incidentally the core business of a company. So to achieve business agility and quickly adapt to an ever changing landscape, companies which rely on software technologies need to be able to quickly adapt their software management process \cite{mathiassen_business_2006}.
